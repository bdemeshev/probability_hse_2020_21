\documentclass[12pt]{article}

\usepackage{tikz} % картинки в tikz
\usepackage{microtype} % свешивание пунктуации

\usepackage{array} % для столбцов фиксированной ширины

\usepackage{indentfirst} % отступ в первом параграфе

\usepackage{sectsty} % для центрирования названий частей
\allsectionsfont{\centering}

\usepackage{amsmath} % куча стандартных математических плюшек

\usepackage{comment}
\usepackage{amsfonts}

\usepackage[top=2cm, left=1cm, right=1cm, bottom=2cm]{geometry} % размер текста на странице

\usepackage{lastpage} % чтобы узнать номер последней страницы

\usepackage{enumitem} % дополнительные плюшки для списков
%  например \begin{enumerate}[resume] позволяет продолжить нумерацию в новом списке
\usepackage{caption}

\usepackage{hyperref} % гиперссылки

\usepackage{multicol} % текст в несколько столбцов


\usepackage{fancyhdr} % весёлые колонтитулы
\pagestyle{fancy}
\lhead{Теория вероятностей-ВШЭ}
\chead{2020-10-12}
\rhead{Контрольная 1. Минимум :)}
\lfoot{}
\cfoot{}
\rfoot{}
\renewcommand{\headrulewidth}{0.4pt}
\renewcommand{\footrulewidth}{0.4pt}



\usepackage{todonotes} % для вставки в документ заметок о том, что осталось сделать
% \todo{Здесь надо коэффициенты исправить}
% \missingfigure{Здесь будет Последний день Помпеи}
% \listoftodos --- печатает все поставленные \todo'шки


% более красивые таблицы
\usepackage{booktabs}
% заповеди из докупентации:
% 1. Не используйте вертикальные линни
% 2. Не используйте двойные линии
% 3. Единицы измерения - в шапку таблицы
% 4. Не сокращайте .1 вместо 0.1
% 5. Повторяющееся значение повторяйте, а не говорите "то же"


\usepackage{fontspec}
\usepackage{polyglossia}

\setmainlanguage{russian}
\setotherlanguages{english}

% download "Linux Libertine" fonts:
% http://www.linuxlibertine.org/index.php?id=91&L=1
\setmainfont{Linux Libertine O} % or Helvetica, Arial, Cambria
% why do we need \newfontfamily:
% http://tex.stackexchange.com/questions/91507/
\newfontfamily{\cyrillicfonttt}{Linux Libertine O}

\AddEnumerateCounter{\asbuk}{\russian@alph}{щ} % для списков с русскими буквами
\setlist[enumerate, 2]{label=\asbuk*),ref=\asbuk*}

%% эконометрические сокращения
\DeclareMathOperator{\Cov}{Cov}
\DeclareMathOperator{\Corr}{Corr}
\DeclareMathOperator{\Var}{Var}
\DeclareMathOperator{\E}{E}
\def \hb{\hat{\beta}}
\def \hs{\hat{\sigma}}
\def \htheta{\hat{\theta}}
\def \s{\sigma}
\def \hy{\hat{y}}
\def \hY{\hat{Y}}
\def \v1{\vec{1}}
\def \e{\varepsilon}
\def \he{\hat{\e}}
\def \z{z}
\def \hVar{\widehat{\Var}}
\def \hCorr{\widehat{\Corr}}
\def \hCov{\widehat{\Cov}}
\def \cN{\mathcal{N}}
\def \P{\mathbb{P}}


\begin{document}

В этот день в 1492 году экспедиция Христофора Колумба достигла острова Сан-Сальвадор. 

Это и есть «открытие Америки».

\begin{enumerate}
\item Дайте определение независимости (попарной и в совокупности) для $n$ случайных событий.
\item Дайте определение функции распределения $F_{X}(x)$ случайной величины $X$. 
Выпишите свойства функции $F_{X}(x)$. 
\item Выпишите формулу Байеса, указав условия её применимости.

\item  Пусть $\P(A) = 0.5, \P(B) = 0.3, \P(A\cap B) = 0.2$.
\begin{enumerate}
  \item  Найдите $\P(A|B)$;
  \item  Найдите $\P(A\cup B)$;
  \item  Являются ли события $A$ и $B$ независимыми?
\end{enumerate}

\item  Пусть случайная величина $X$ имеет таблицу распределения: % задача 10

\begin{tabular}{lccc}
\toprule
$x$ & $-1$  & $0$  & $3$ \\
$\P(X = x)$ & $0.2$ & $0.3$  & $0.5$ \\
\bottomrule
\end{tabular}

Найдите
\begin{enumerate}
	\item $\E(X)$
	\item $\E\left(X^2\right)$
	\item $\P(X=0 \mid X \geq 0)$, $\E(X \mid X \geq 0)$
\end{enumerate}

\end{enumerate}

\newpage

В этот день в 1492 году экспедиция Христофора Колумба достигла острова Сан-Сальвадор. 

Это и есть «открытие Америки».

\begin{enumerate}
\item Дайте определение независимости (попарной и в совокупности) для $n$ случайных событий.
\item Дайте определение функции распределения $F_{X}(x)$ случайной величины $X$. 
Выпишите свойства функции $F_{X}(x)$. 
\item Выпишите формулу Байеса, указав условия её применимости.

\item  Пусть $\P(A) = 0.2, \P(B) = 0.4, \P(A\cap B) = 0.3$.
\begin{enumerate}
  \item  Найдите $\P(A|B)$;
  \item  Найдите $\P(A\cup B)$;
  \item  Являются ли события $A$ и $B$ независимыми?
\end{enumerate}

\item  Пусть случайная величина $X$ имеет таблицу распределения: % задача 10

\begin{tabular}{lccc}
\toprule
$x$ & $-1$  & $0$  & $3$ \\
$\P(X = x)$ & $0.3$ & $0.4$  & $0.3$ \\
\bottomrule
\end{tabular}

Найдите
\begin{enumerate}
	\item $\E(X)$
	\item $\E\left(X^2\right)$
	\item $\P(X=0 \mid X \geq 0)$, $\E(X \mid X \geq 0)$
\end{enumerate}

\end{enumerate}



\end{document}
