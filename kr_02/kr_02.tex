\documentclass[12pt]{article}

\usepackage{tikz} % картинки в tikz
\usepackage{microtype} % свешивание пунктуации

\usepackage{array} % для столбцов фиксированной ширины

\usepackage{indentfirst} % отступ в первом параграфе

\usepackage{sectsty} % для центрирования названий частей
\allsectionsfont{\centering}

\usepackage{amsmath} % куча стандартных математических плюшек

\usepackage{comment}
\usepackage{amsfonts}

\usepackage[top=2cm, left=1cm, right=1cm, bottom=2cm]{geometry} % размер текста на странице

\usepackage{lastpage} % чтобы узнать номер последней страницы

\usepackage{enumitem} % дополнительные плюшки для списков
%  например \begin{enumerate}[resume] позволяет продолжить нумерацию в новом списке
\usepackage{caption}

\usepackage{hyperref} % гиперссылки

\usepackage{multicol} % текст в несколько столбцов


\usepackage{fancyhdr} % весёлые колонтитулы
\pagestyle{fancy}
\lhead{Теория вероятностей-ВШЭ}
\chead{2020-12-12}
\rhead{Контрольная 2. Максимум :)}
\lfoot{}
\cfoot{}
\rfoot{}
\renewcommand{\headrulewidth}{0.4pt}
\renewcommand{\footrulewidth}{0.4pt}



\usepackage{todonotes} % для вставки в документ заметок о том, что осталось сделать
% \todo{Здесь надо коэффициенты исправить}
% \missingfigure{Здесь будет Последний день Помпеи}
% \listoftodos --- печатает все поставленные \todo'шки


% более красивые таблицы
\usepackage{booktabs}
% заповеди из докупентации:
% 1. Не используйте вертикальные линни
% 2. Не используйте двойные линии
% 3. Единицы измерения - в шапку таблицы
% 4. Не сокращайте .1 вместо 0.1
% 5. Повторяющееся значение повторяйте, а не говорите "то же"


\usepackage{fontspec}
\usepackage{polyglossia}

\setmainlanguage{russian}
\setotherlanguages{english}

% download "Linux Libertine" fonts:
% http://www.linuxlibertine.org/index.php?id=91&L=1
\setmainfont{Linux Libertine O} % or Helvetica, Arial, Cambria
% why do we need \newfontfamily:
% http://tex.stackexchange.com/questions/91507/
\newfontfamily{\cyrillicfonttt}{Linux Libertine O}

\AddEnumerateCounter{\asbuk}{\russian@alph}{щ} % для списков с русскими буквами
\setlist[enumerate, 2]{label=\asbuk*),ref=\asbuk*}
%\setlist[enumerate, 1]{label=\asbuk*),ref=\asbuk*}


%% эконометрические сокращения
\DeclareMathOperator{\Cov}{Cov}
\DeclareMathOperator*{\plim}{plim}
\DeclareMathOperator{\Corr}{Corr}
\DeclareMathOperator{\Var}{Var}
\DeclareMathOperator{\E}{E}
\def \hb{\hat{\beta}}
\def \hs{\hat{\sigma}}
\def \htheta{\hat{\theta}}
\def \s{\sigma}
\def \hy{\hat{y}}
\def \hY{\hat{Y}}
\def \v1{\vec{1}}
\def \e{\varepsilon}
\def \he{\hat{\e}}
\def \z{z}
\def \hVar{\widehat{\Var}}
\def \hCorr{\widehat{\Corr}}
\def \hCov{\widehat{\Cov}}
\def \cN{\mathcal{N}}
\def \P{\mathbb{P}}


\begin{document}

Вариация 1

\begin{enumerate}
\item Пусть $(X_n)_{n=1}^{\infty}$ — последовательность независимых случайных величин, равномерно распределённых на интервале $(0; 2)$. 
Найдите пределы:

\begin{enumerate}
    \item (2 балла) 
    \[
        \plim\limits_{n\to\infty} \frac{X_n}{n};
    \]
    \item (2 балла) 
    \[ 
        \plim\limits_{n\to\infty} \frac{X_1 + X_2 + \ldots + X_{10}}{n};
    \]
    \item (2 балла) 
    \[
         \plim\limits_{n\to\infty} \left(\frac{X_1 + X_2 + \ldots + X_n}{n}\right)^2;
         \]
    
    \item (4 балла) 
    \[
         \plim\limits_{n\to\infty} \left(\frac{X_1^2 + X_2^2 + \ldots + X_{n}^2}{n} - \frac{X_1 + X_2 + \ldots + X_{n}}{n}\right).
         \]
\end{enumerate}

\item Известно, что кандидата в президенты от оппозиции поддерживает в среднем $p\cdot 100\%$ избирателей. 
На некотором избирательном участке проголосовало 100 избирателей. 
По результатам голосований на этом участке была рассчитана выборочная доля проголосовавших 
за оппозиционного кандидата (от общего числа проголосовавших на участке).

\begin{enumerate}
    \item (2 балла) С помощью неравенства Маркова оцените сверху вероятность того, 
    что выборочная доля окажется выше 0.5, если известно, что $p<0.1$.
    \item (3 балла) С помощью неравенства Чебышева оцените снизу вероятность того, 
    что модуль отклонения выборочной доли от истинной будет меньше $0.1$.
    \item (2 балла) С помощью центральной предельной теоремы приближенно найдите вероятность того, 
    что выборочная доля окажется больше 0.3, если известно, что $p=0.1$.
    \item (3 балла) С помощью неравенства Берри-Эссеена оцените погрешность приближённого значения вероятности из пункта (в).
    \item (5 баллов) С помощью центральной предельной теоремы оцените снизу вероятность из пункта (б) для неизвестной $p$.
\end{enumerate}

\end{enumerate}


\newpage


Вариация 2

\begin{enumerate}
\item Пусть $(X_n)_{n=1}^{\infty}$ — последовательность независимых случайных величин, 
равномерно распределённых на интервале $(0; 3)$. 


Найдите пределы:

\begin{enumerate}
    \item (2 балла) 
    \[
        \plim\limits_{n\to\infty} \frac{X_n}{n};
    \]
    \item (2 балла) 
    \[ 
        \plim\limits_{n\to\infty} \frac{X_1 + X_2 + \ldots + X_{10}}{n};
    \]
    \item (2 балла) 
    \[
         \plim\limits_{n\to\infty} \left(\frac{X_1 + X_2 + \ldots + X_n}{n}\right)^2;
         \]
    
    \item (4 балла) 
    \[
         \plim\limits_{n\to\infty} \left(\frac{X_1^2 + X_2^2 + \ldots + X_{n}^2}{n} - \frac{X_1 + X_2 + \ldots + X_{n}}{n}\right).
         \]
\end{enumerate}

\item Известно, что кандидата в президенты от оппозиции поддерживает в среднем $p\cdot 100\%$ избирателей. 
На некотором избирательном участке проголосовало 100 избирателей. 
По результатам голосований на этом участке была рассчитана выборочная доля проголосовавших 
за оппозиционного кандидата (от общего числа проголосовавших на участке).

\begin{enumerate}
    \item (2 балла) С помощью неравенства Маркова оцените сверху вероятность того, 
    что выборочная доля окажется выше 0.5, если известно, что $p<0.15$.
    \item (3 балла) С помощью неравенства Чебышева оцените снизу вероятность того, 
    что модуль отклонения выборочной доли от истинной будет меньше $0.15$.
    \item (2 балла) С помощью центральной предельной теоремы приближенно найдите вероятность того, 
    что выборочная доля окажется больше 0.3, если известно, что $p=0.15$.
    \item (3 балла) С помощью неравенства Берри-Эссеена оцените погрешность приближённого значения вероятности из пункта (в).
    \item (5 баллов) С помощью центральной предельной теоремы оцените снизу вероятность из пункта (б) для неизвестной $p$.
\end{enumerate}

\end{enumerate}


\newpage


Вариация 3

\begin{enumerate}
\item Пусть $(X_n)_{n=1}^{\infty}$ — последовательность независимых случайных величин, равномерно распределённых на интервале $(0; 4)$. 
Найдите пределы:

\begin{enumerate}
    \item (2 балла) 
    \[
        \plim\limits_{n\to\infty} \frac{X_n}{n};
    \]
    \item (2 балла) 
    \[ 
        \plim\limits_{n\to\infty} \frac{X_1 + X_2 + \ldots + X_{10}}{n};
    \]
    \item (2 балла) 
    \[
         \plim\limits_{n\to\infty} \left(\frac{X_1 + X_2 + \ldots + X_n}{n}\right)^2;
         \]
    
    \item (4 балла) 
    \[
         \plim\limits_{n\to\infty} \left(\frac{X_1^2 + X_2^2 + \ldots + X_{n}^2}{n} - \frac{X_1 + X_2 + \ldots + X_{n}}{n}\right).
         \]
\end{enumerate}

\item Известно, что кандидата в президенты от оппозиции поддерживает в среднем $p\cdot 100\%$ избирателей. 
На некотором избирательном участке проголосовало 100 избирателей. 
По результатам голосований на этом участке была рассчитана выборочная доля проголосовавших 
за оппозиционного кандидата (от общего числа проголосовавших на участке).

\begin{enumerate}
    \item (2 балла) С помощью неравенства Маркова оцените сверху вероятность того, 
    что выборочная доля окажется выше 0.5, если известно, что $p<0.2$.
    \item (3 балла) С помощью неравенства Чебышева оцените снизу вероятность того, 
    что модуль отклонения выборочной доли от истинной будет меньше $0.2$.
    \item (2 балла) С помощью центральной предельной теоремы приближенно найдите вероятность того, 
    что выборочная доля окажется больше 0.3, если известно, что $p=0.2$.
    \item (3 балла) С помощью неравенства Берри-Эссеена оцените погрешность приближённого значения вероятности из пункта (в).
    \item (5 баллов) С помощью центральной предельной теоремы оцените снизу вероятность из пункта (б) для неизвестной $p$.
\end{enumerate}

\end{enumerate}


\newpage


Вариация 4

\begin{enumerate}
\item Пусть $(X_n)_{n=1}^{\infty}$ — последовательность независимых случайных величин, равномерно распределённых на интервале $(0; 5)$. 
Найдите пределы:

\begin{enumerate}
    \item (2 балла) 
    \[
        \plim\limits_{n\to\infty} \frac{X_n}{n};
    \]
    \item (2 балла) 
    \[ 
        \plim\limits_{n\to\infty} \frac{X_1 + X_2 + \ldots + X_{10}}{n};
    \]
    \item (2 балла) 
    \[
         \plim\limits_{n\to\infty} \left(\frac{X_1 + X_2 + \ldots + X_n}{n}\right)^2;
         \]
    
    \item (4 балла) 
    \[
         \plim\limits_{n\to\infty} \left(\frac{X_1^2 + X_2^2 + \ldots + X_{n}^2}{n} - \frac{X_1 + X_2 + \ldots + X_{n}}{n}\right).
         \]
\end{enumerate}

\item Известно, что кандидата в президенты от оппозиции поддерживает в среднем $p\cdot 100\%$ избирателей. 
На некотором избирательном участке проголосовало 100 избирателей. 
По результатам голосований на этом участке была рассчитана выборочная доля проголосовавших 
за оппозиционного кандидата (от общего числа проголосовавших на участке).

\begin{enumerate}
    \item (2 балла) С помощью неравенства Маркова оцените сверху вероятность того, 
    что выборочная доля окажется выше 0.5, если известно, что $p<0.25$.
    \item (3 балла) С помощью неравенства Чебышева оцените снизу вероятность того, 
    что модуль отклонения выборочной доли от истинной будет меньше $0.25$.
    \item (2 балла) С помощью центральной предельной теоремы приближенно найдите вероятность того, 
    что выборочная доля окажется больше 0.3, если известно, что $p=0.25$.
    \item (3 балла) С помощью неравенства Берри-Эссеена оцените погрешность приближённого значения вероятности из пункта (в).
    \item (5 баллов) С помощью центральной предельной теоремы оцените снизу вероятность из пункта (б) для неизвестной $p$.
\end{enumerate}

\end{enumerate}







\end{document}
