\documentclass[12pt]{article}

\usepackage{tikz} % картинки в tikz
\usepackage{microtype} % свешивание пунктуации

\usepackage{array} % для столбцов фиксированной ширины

\usepackage{indentfirst} % отступ в первом параграфе

\usepackage{sectsty} % для центрирования названий частей
\allsectionsfont{\centering}

\usepackage{amsmath} % куча стандартных математических плюшек

\usepackage{comment}
\usepackage{amsfonts}

\usepackage[top=2cm, left=1cm, right=1cm, bottom=2cm]{geometry} % размер текста на странице

\usepackage{lastpage} % чтобы узнать номер последней страницы

\usepackage{enumitem} % дополнительные плюшки для списков
%  например \begin{enumerate}[resume] позволяет продолжить нумерацию в новом списке
\usepackage{caption}

\usepackage{hyperref} % гиперссылки

\usepackage{multicol} % текст в несколько столбцов


\usepackage{fancyhdr} % весёлые колонтитулы
\pagestyle{fancy}
\lhead{Теория вероятностей-ВШЭ}
\chead{2020-10-22}
\rhead{Контрольная 1. Максимум :)}
\lfoot{}
\cfoot{}
\rfoot{}
\renewcommand{\headrulewidth}{0.4pt}
\renewcommand{\footrulewidth}{0.4pt}



\usepackage{todonotes} % для вставки в документ заметок о том, что осталось сделать
% \todo{Здесь надо коэффициенты исправить}
% \missingfigure{Здесь будет Последний день Помпеи}
% \listoftodos --- печатает все поставленные \todo'шки


% более красивые таблицы
\usepackage{booktabs}
% заповеди из докупентации:
% 1. Не используйте вертикальные линни
% 2. Не используйте двойные линии
% 3. Единицы измерения - в шапку таблицы
% 4. Не сокращайте .1 вместо 0.1
% 5. Повторяющееся значение повторяйте, а не говорите "то же"


\usepackage{fontspec}
\usepackage{polyglossia}

\setmainlanguage{russian}
\setotherlanguages{english}

% download "Linux Libertine" fonts:
% http://www.linuxlibertine.org/index.php?id=91&L=1
\setmainfont{Linux Libertine O} % or Helvetica, Arial, Cambria
% why do we need \newfontfamily:
% http://tex.stackexchange.com/questions/91507/
\newfontfamily{\cyrillicfonttt}{Linux Libertine O}

\AddEnumerateCounter{\asbuk}{\russian@alph}{щ} % для списков с русскими буквами
\setlist[enumerate, 2]{label=\asbuk*),ref=\asbuk*}
%\setlist[enumerate, 1]{label=\asbuk*),ref=\asbuk*}


%% эконометрические сокращения
\DeclareMathOperator{\Cov}{Cov}
\DeclareMathOperator{\Corr}{Corr}
\DeclareMathOperator{\Var}{Var}
\DeclareMathOperator{\E}{E}
\def \hb{\hat{\beta}}
\def \hs{\hat{\sigma}}
\def \htheta{\hat{\theta}}
\def \s{\sigma}
\def \hy{\hat{y}}
\def \hY{\hat{Y}}
\def \v1{\vec{1}}
\def \e{\varepsilon}
\def \he{\hat{\e}}
\def \z{z}
\def \hVar{\widehat{\Var}}
\def \hCorr{\widehat{\Corr}}
\def \hCov{\widehat{\Cov}}
\def \cN{\mathcal{N}}
\def \P{\mathbb{P}}


\begin{document}

Блок 1. Задачи для БП + ИП. Время написания: 50 минут. 

\begin{enumerate}
\item Подбросили две игральные кости. 
Введем следующие события: $A$ — на первой кости выпала тройка, 
$B$ — сумма очков является четным числом, $C$ — на второй кости выпало больше очков, чем на первой.

\begin{enumerate}

    \item Найдите вероятность каждого из событий $A$, $B$ и  $C$.

    \item Найдите условную вероятность $\mathbb{P}\{A|C\}$.

    \item Проверьте, будут ли события $A$, $C$ и $B \cap C$ попарно независимыми.

\end{enumerate}

\item Известно, что случайная величина $\xi$ имеет плотность распределения

\[f_{\xi}(x) = 
\begin{cases}
\dfrac{-x^2}{36} + \dfrac{1}{4}, \text{ если } x \in [-3; 3], \\
0, \text{ иначе}.
\end{cases}
\]

    \begin{enumerate}

    \item Найдите $\mathbb{P}\{\xi \in [0; 2]\}$.

    \item Найдите $\mathbb{P}\{\xi \in [2; 4]\}$.

    \item Найдите $\mathbb{P}\{\xi = 1.5\}$.

    \item Найдите $\mathbb{E}[\xi]$.

    \item Найдите $\mathbb{E}[\xi^2]$.

\end{enumerate}


\item Пусть $\Omega = \{a, \, b, \, c, \, d\}$ — пространство элементарных событий. 
Рассмотрим систему множеств $\mathcal{G} = \bigl\{ \{a, \, b, \, c\}, \, \{c, \, d\}\bigr\}$.

\begin{enumerate}

     \item Объясните, почему система $\mathcal{G}$ не является $\sigma$-алгеброй.

    \item Добавим к $\mathcal{G}$ множество $\{c\}$. 
    В результате мы получим $\mathcal{H} = \bigl\{ \{a, \, b, \, c\}, \, \{c, \, d\}, \, \{c\}\bigr\}$. 
    Является ли система $\mathcal{H}$ $\sigma$-алгеброй? Почему?

    \item Если $\mathcal{H}$ не является $\sigma$-алгеброй, 
    то дополните систему $\mathcal{H}$ множествами так, чтобы она стала $\sigma$-алгеброй.

    \item Бонусный пункт. Случайная величина $\xi$ имеет функцию плотности
    $f_{\xi}(x) = 
    \dfrac{1}{\pi(5+4x+x^2)}
    $, причем $x \in \mathbb{R}$. 
    Если возможно, найдите моду, медиану и математическое ожидание $\xi$.
\end{enumerate}

\end{enumerate}




\newpage

Блок 2. Задачи для БП + ИП. Время написания 60 минут. 

\begin{enumerate}
\item \textbf{Основано на реальных событиях.} Муж и жена заболели коронавирусом и поправились. 

Для выписки необходимо, чтобы два подряд проведенных анализа на коронавирус у каждого из них оказались отрицательными. 
Результаты анализа известны на третий день после сдачи. 
Супруги заинтересованы в том, чтобы как можно скорее выйти на работу. 

Если первые анализы у обоих отрицательные, на следующий день они сдают анализ повторно, 
если хотя бы у одного из них анализ положительный, им приходится сидеть на карантине 2 недели, 
после чего алгоритм выписки повторяется. 
Известно, что чувствительность тестов (вероятность отрицательного результата, если человек не болен) 
составляет 0.9. 
	
\begin{enumerate}
    
    \item Найдите вероятность того, что супругам удастся выписаться не раньше, чем через месяц;
    
    \item Найдите математическое ожидание дней до выписки, если считать с дня первого теста.
    
\end{enumerate}

\item  В лифт 9-этажного дома на первом этаже вошли 5 человек. 
Они выходят на каждом этаже начиная со второго равновероятно и независимо друг от друга.

\begin{enumerate}
    
    \item Найдите вероятность того, что с третьего по шестой этаж не выйдет ни один пассажир.
    
    \item Найдите ожидаемое число пассажиров, которые выйдут с третьего по шестой этаж.
    
    \item Найдите дисперсию числа пассажиров, которые выйдут с третьего по шестой этаж.
    
    \item Найдите наиболее вероятное число пассажиров, которые выйдут с третьего по шестой этаж.
    
\end{enumerate}

\item Известно, что случайная величина $\xi$ принимает значения на отрезке $[0; 1]$. 
Для любых точек $0 \leq a \leq b \leq 1$ 
вероятности описываются формулой $\mathbb{P}{\{a \leq \xi \leq b\}} = (b+a) \cdot (b - a)$. 

Найдите плотность случайной величины $\xi$.
\end{enumerate}

\newpage
Блок 2. Задачи для ИП. Необходимо решить две из четырех задач. Время написания 140 минут. Выдаются одновременно с общими задачами.

\begin{enumerate}[resume]
    \item В 1786 году Лаплас для оценки числа $N$ жителей Франции предложил следующий метод. 
    Выберем некоторое число, скажем, $M$, элементов популяции и пометим их. 
    Затем возвратим их в основную совокупность и предположим, что они «хорошо перемешаны» 
    с немаркированными элементами. Возьмем из «перемешанной» популяции $n$ элементов. 
    Обозначим через $X$ число маркированных элементов в этой выборке из $n$ элементов.

    \begin{enumerate}
        \item Найдите распределение случайной величины $X$.
        \item Полагая $M$, $n$ и $m$ заданными, найдите наиболее правдоподобный объём всей популяции – значение $N$, 
        дающее наибольшую вероятность получить число маркированных элементов, равное $m$.
    \end{enumerate}

    \item Чеканщик визиря долгие годы потихонечку скромно подделывал золотые монеты. 
    Из ста монет одну делал из меди, а не из золота, а золото забирал себе. 
    После доноса на чеканщика визирь решил проверить, действительно ли чеканщик подделывает монеты.

    Для этого каждый день решено проверять один мешок из 1000 монет: 
    выбирать случайным образом 50 монет и проверять, есть ли среди них фальшивые. 
    Если будет обнаружена одна фальшивая монета, то на первый раз чеканщик прощается, 
    и на следующий день проверка продолжается. При повторном обнаружении фальшивой монеты чеканщика казнят. Если при проверке будут обнаружены сразу две и более фальшивых монеты, чеканщика в тот же день казнят.

    Число мешков бесконечно.

    \begin{enumerate}
        \item Найдите распределение количества дней жизни чеканщика $T$, начиная с первого дня проверки.
        \item  Найдите математическое ожидание $T$.
    \end{enumerate}

    \item Андрей, Белла, Вера и Гриша — студенты. 
    Любые два студента знакомы друг с другом с вероятностью $p$ независимо от других.
    Только что Андрей прочитал новую задачку по теории вероятностей. Если студент узнал о новой задачке, то он обязательно поделится ею со всеми знакомыми.

    \begin{enumerate}
        \item Какова вероятность того, что Гриша узнает о новой задачке?
        \item Какова вероятность того, что Гриша узнает о новой задачке, если Вера не узнала о ней?
    \end{enumerate}
    
    \item Ровно 40 лет назад, 22 октября 1980 года папа римский отменил вердикт, осуждающий Галилея!


    Как всем известно, Земля — плоская и имеет форму круга радиуса 1. 
    Папа римский и Галилей равномерно и независимо выбирают две точки на краю Земли.

    \begin{enumerate}
        \item Найдите функцию плотности расстояния от папы римского до Галилея.
        \item Найдите функцию плотности квадрата расстояния.
    \end{enumerate}
    

    

    

\end{enumerate}





\end{document}
